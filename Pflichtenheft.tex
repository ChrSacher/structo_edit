\documentclass[a4paper,10pt]{report}
\usepackage[utf8]{inputenc}
\usepackage{color}
\usepackage{colortbl}

% Title Page
%\title{}
%\author{}

\definecolor{Gray}{gray}{0.8}
\definecolor{lightGray}{gray}{0.925}

\newcolumntype{A}{>{\columncolor{Gray}}c}
\newcolumntype{B}{>{\columncolor{Gray}}l}

\begin{document}
%\maketitle
\begin{titlepage}

\begin{center}
\begin{tabular}{|A|}
\hline
\\
\\
\\
\bfseries \Huge \quad \quad \quad  Pflichtenheft   \quad \quad \quad
\\
\\
\Large zum Softwareprojekt\\
\Large (Prof. Steinbach)\\
\\
\\
\hline
\end{tabular}
\end{center}

\vspace{3,5mm}

\begin{center}
\begin{tabular}{|A|}
%Hier Projektnamen einfügen.
%Bei Bedarf die \quad entfernen, sie dienen nur der einheitlichen Breite der Tabellen
\hline
\\
\\
\bfseries \Large \quad \quad \quad Projektname (Projektnummer) \quad \quad \quad 
\\
\\
\\
\\
\hline
\end{tabular}
\end{center}

\vspace{9mm}

\bfseries \large Angaben zu den am Projekt beteiligten Studenten:

\begin{center}
\begin{tabular}{|c|l|l|l|l|}
%Hier Mitglieder der Projektgruppe eintragen. Vor jeden Namen ein \normalsize setzen; wird die Tabelle durch lange Namen zu groß \normalsize durch \small ersetzen
\hline
\rowcolor{Gray}\normalsize &\normalsize Name, Vorname &\normalsize Mat.-Nr. &\normalsize Studiengang &\normalsize Email-Adresse \\
\hline
\rowcolor{lightGray}\normalsize 1. & & & & \\
\hline
\rowcolor{Gray}\normalsize 2. & & & & \\
\hline
\rowcolor{lightGray}\normalsize 3. & & & & \\
\hline
\rowcolor{Gray}\normalsize 4. & & & & \\
\hline
\rowcolor{lightGray}\normalsize 5. & & & & \\
\hline
\end{tabular}
\end{center}

\vspace{10mm}

\begin{center}
\begin{tabular}{|BB|}
\hline
\bfseries \large Best\"{a}tigt durch Prof. Steinbach & \quad \quad \quad \quad \quad \quad \quad \quad \quad \\
\bfseries \large Datum, Unterschrift &
\\
\hline
\end{tabular}
\end{center}

\end{titlepage}
%\begin{abstract}
%\end{abstract}

\newpage
\tableofcontents 
\newpage
\section{Zielbestimmung}
\subsection{Musskriterien}
\begin{itemize}
\item Struktogramm dynamisch erstellen
\item GUI zur Benutzerfreundlichen Bedienung
\item Baumstruktur des Struktogramms visualisieren
\item Speichern und Laden von Struktogrammen
\item 
\end{itemize}
\subsection{Wuschkriteien}
\begin{itemize}
\item XML Generierung aus Struktogrammen. Dies soll zum vereinfachten exportieren dienen geht und geht damit mit speichern und laden einher.
\item Visualisierung des ablaufen des Programmes, welches im Struktogramm vorliegt. Quasi als "Programmoutput"
\item 
\end{itemize}
\subsection{Abgrenzungskriterien}
\begin{itemize}
\item Es soll kein funktionierendes Programm aus dem Struktogramm generiert werden.
\end{itemize}

\section{Produkteinsatz}
\subsection{Anwendungsbereiche}
Das Programm soll Leuten helfen, die neu ins Programmieren oder in die Informatik einsteigen. Aber vor allem soll es das algorithmische Denken veranschaulichen.
\subsection{Zielgruppen}
\begin{itemize}
\item Schüler und Studenten
\item Informatik Menschen...
\end{itemize}
\subsection{Betirebsbedingungen}

\section{Produktumgebung}
\subsection{Software}
\subsection{Hardware}
\subsection{Orgware}
\subsection{Produktschnittstellen}

\section{Produktfunktionen}
\begin{itemize}
\item /F10/ Neuen logischen Block erstellen
\item /F20/ 
\end{itemize}

\section{Produktdaten}
\section{Produktleistungen}
\section{Benutzeroberfläche}
\section{Qualitätszielbestimmung}
\section{Globale Testszenarien/Testfälle}
\section{Entwicklungsumgebung}
\section{Ergänzungen}
\section{Verteilung der Aufgaben zwischen den Projektteilnehmer}

\end{document}          

